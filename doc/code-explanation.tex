\chapter{Υλοποίηση σε CUDA}

Υλοποιήθηκαν οι εξής εκδόσεις του προγράμματος:

\begin{enumerate}
\item Σειριακός κώδικας (single/cpu1.c)
\item Κώδικας σε Openmp μαζί με διάφορα micro-optimizations (single/omp.c).
\item Cuda ένα κελί ανά νήμα (single/cuda-single.cu).
\item Cuda Πολλαπλά κελιά ανά νήμα: υλοποίηση σε bitboard 32-bit (bboard2d).
\item Cuda Πολλαπλά κελιά ανά νήμα: υλοποίηση σε bitboard 32-bit με shared memory (bboard2d-shared).
\item Cuda Πολλαπλά κελιά ανά νήμα: υλοποίηση σε bitfields 64-bit (bboard1x64).
\item Cuda Πολλαπλά κελιά ανά νήμα: υλοποίηση σε bitfields 64-bit με shared memory (bboard1x64-shared).
\end{enumerate}

Η υλοποίηση της σειριακής, openmp και ένα κελί ανά νήμα υλοποίησης είναι αρκετά απλή και δεν
περιγράφεται. Οι 4 τελευταίες εκδόσεις λειτουργούν με παρόμοιο τρόπο αλλά
η ανάπτυξή τους διαχωρίστηκε από ένα σημείο και μετά
και για αυτό αντιμετωπίζονται σαν ξεχωριστά project
(διαφορετικό CMakelists.txt, διαφορετικός φάκελος κτλπ).

Τα bitboards επιλέχτηκαν καθώς ελαττώνουν την χρήση της global μνήμης σύμφωνα με τον μέγεθός τους
και επιτρέπουν την αποφυγή for και if block μέσω της χρήσης bitwise πράξεων.

\begin{sloppypar}
Η μετατροπή από κανονικό σε "tiled" πίνακα γίνεται στα αρχεία converters.cu
με την βοήθεια των συναρτήσεων \lstinline!convert_from_tiled()!
και \lstinline!convert_to_tiled()!.
Ο τύπος των δεδομένων σε tiled μορφή είναι \lstinline!bboard!
και στα αρχεία utils.h ορίζονται διάφορα macro για την προσπέλαση των bits.
Η πρόσβαση στην global μνήμη της συσκευής γίνεται μέσω της χρήσης της \lstinline!cudaMallocPitch()!
που θεωρείται καλύτερη για την πρόσβαση σε δισδιάστατους πίνακες.
Στην global μνήμη αποθηκεύονται δύο πίνακες,
\lstinline!bboard* d_board! και
\lstinline!bboard* d_help!.
Ο πρώτος κρατάει το αποτέλεσμα της τρέχουσας γενιάς
και ο δεύτερος χρησιμοποιείται για τον υπολογισμό της επόμενης.
\end{sloppypar}

\begin{sloppypar}
Ο υπολογισμός της κάθε γενιάς γίνεται στα αρχεία calculator.cu μέσω της \lstinline!calculate_next_generation()!.
\end{sloppypar}

\section{bboard2d}
\begin{sloppypar}
Στον πίνακα \lstinline!bboard neighbors[3][3]! αποθηκεύονται τα γειτονικά tiles του κάθε thread.
Έτσι, στην θέση [1][1] είναι τα τοπικό tile,
στην θέση [0][0] το πάνω αριστερά,
στην θέση [2][1] το κάτω κ.ο.κ.
\end{sloppypar}
\begin{sloppypar}
Στην έκδοση χωρίς χρήση shared memory το κάθε thread γεμίζει τα 9 κελιά του \lstinline!neighbors! με απευθείας πρόσβαση στην global μνήμη.
Στην έκδοση με shared memory, το κάθε thread γεμίζει μια θέση που του αναλογεί από τον πίνακα
\lstinline!__shared__ bboard tiles[]! και
τα threads που βρίσκονται στα σύνορα του block αναλαμβάνουν να γεμίσουν τις θέσεις που
αντιστοιχούν στα tiles με τα οποία συνορεύει το block.
Τέλος, μετά την \lstinline!__syncthreads();! οι θέσεις του \lstinline!neighbors! γεμίζονται
μέσω της πρόσβασης στον \lstinline!tiles!.
\end{sloppypar}

\begin{sloppypar}
Εδώ, υπάρχει μια διαφοροποίηση ανάλογα με το αν η διάσταση $N$ του πίνακα διαιρείται τέλεια με το 8.
Όταν αυτό ισχύει χρησιμοποιείται η συνάρτηση \lstinline!calculate_next_generation_no_rem()!
από την \lstinline!main()! αλλιώς η \lstinline!calculate_next_generation()!.
Αυτό συμβαίνει γιατί η μέθοδος με τις bitwise πράξεις της πρώτης είναι δυσκολότερο να εφαρμοστεί και στην δεύτερη.
\end{sloppypar}

Στην \lstinline!calculate_next_generation()! ο υπολογισμός γίνεται με απλή for στις διαστάσεις
\lstinline!WIDTH! και \lstinline!HEIGHT! του bitboard.
Εφαρμόζονται μερικές μικρο-βελτιώσεις με την χρήση του preprocessor ώστε να μεταφερθούν οι
οριακές περιπτώσεις εκτός της for και να αποφευχθούν μερικές if.

Στην \lstinline!calculate_next_generation_no_rem()! αρχικά εφαρμόζονται μάσκες και bitwise
πράξει στα στοιχεία του πίνακα \lstinline!neighbors! ώστε να ενωθούν σε ένα μεγαλύτερο bitboard τύπου \lstinline!ext_bboard!. 
Για παράδειγμα αν έχουμε τον εξής πίνακα \lstinline!neighbors!:\\
\begin{tabular}{ | c | c | c | c | c | c | c | c |}
\hline
1 & 1 & 1 & 0 & 0 & 1 & 0 & 1 \\ \hline
1 & 0 & 1 & 0 & 1 & 1 & 0 & 1 \\ \hline
1 & 0 & 1 & 0 & 0 & 0 & 1 & 0 \\ \hline
1 & 1 & 1 & 0 & 1 & 1 & 1 & \cellcolor{blue!25}0 \\ \hline
\end{tabular} \
\begin{tabular}{ | c | c | c | c | c | c | c | c |}
\hline
1 & 1 & 1 & 1 & 0 & 0 & 1 & 0 \\ \hline
1 & 1 & 1 & 0 & 0 & 0 & 1 & 0 \\ \hline
1 & 0 & 1 & 0 & 1 & 0 & 1 & 0 \\ \hline
\cellcolor{blue!25}1 & \cellcolor{blue!25}0 & \cellcolor{blue!25}0 & \cellcolor{blue!25}1 & \cellcolor{blue!25}0 & \cellcolor{blue!25}0 & \cellcolor{blue!25}1 & \cellcolor{blue!25}0 \\ \hline
\end{tabular} \
\begin{tabular}{ | c | c | c | c | c | c | c | c |}
\hline
0 & 0 & 1 & 1 & 1 & 0 & 0 & 0 \\ \hline
1 & 1 & 0 & 1 & 0 & 0 & 0 & 1 \\ \hline
0 & 1 & 0 & 0 & 1 & 0 & 1 & 1 \\ \hline
\cellcolor{blue!25}1 & 0 & 1 & 0 & 1 & 0 & 0 & 1 \\ \hline
\end{tabular} \\
\begin{tabular}{ | c | c | c | c | c | c | c | c |}
\hline
1 & 0 & 1 & 0 & 0 & 0 & 0 & \cellcolor{blue!25}1 \\ \hline
0 & 0 & 0 & 0 & 0 & 1 & 1 & \cellcolor{blue!25}1 \\ \hline
0 & 1 & 1 & 0 & 1 & 1 & 1 & \cellcolor{blue!25}1 \\ \hline
1 & 1 & 1 & 0 & 0 & 1 & 1 & \cellcolor{blue!25}0 \\ \hline
\end{tabular} \
\begin{tabular}{ | c | c | c | c | c | c | c | c |}
\hline
1 & 1 & 0 & 1 & 0 & 1 & 0 & 1 \\ \hline
1 & 0 & 0 & 0 & 1 & 1 & 1 & 0 \\ \hline
0 & 1 & 1 & 1 & 1 & 0 & 1 & 0 \\ \hline
1 & 0 & 1 & 0 & 1 & 1 & 0 & 1 \\ \hline
\end{tabular} \
\begin{tabular}{ | c | c | c | c | c | c | c | c |}
\hline
\cellcolor{blue!25}1 & 0 & 0 & 1 & 1 & 0 & 1 & 1 \\ \hline
\cellcolor{blue!25}0 & 0 & 0 & 0 & 0 & 0 & 0 & 1 \\ \hline
\cellcolor{blue!25}0 & 1 & 1 & 1 & 0 & 0 & 1 & 0 \\ \hline
\cellcolor{blue!25}0 & 0 & 0 & 1 & 1 & 1 & 1 & 0 \\ \hline
\end{tabular} \\
\begin{tabular}{ | c | c | c | c | c | c | c | c |}
\hline
0 & 1 & 1 & 1 & 1 & 1 & 1 & \cellcolor{blue!25}1 \\ \hline
1 & 1 & 1 & 1 & 1 & 0 & 1 & 1 \\ \hline
1 & 1 & 0 & 1 & 1 & 0 & 1 & 0 \\ \hline
1 & 0 & 0 & 0 & 0 & 0 & 0 & 1 \\ \hline
\end{tabular} \
\begin{tabular}{ | c | c | c | c | c | c | c | c |}
\hline
\cellcolor{blue!25}1 & \cellcolor{blue!25}0 & \cellcolor{blue!25}1 & \cellcolor{blue!25}0 & \cellcolor{blue!25}1 & \cellcolor{blue!25}1 & \cellcolor{blue!25}1 & \cellcolor{blue!25}1 \\ \hline
1 & 0 & 0 & 0 & 0 & 0 & 0 & 1 \\ \hline
1 & 1 & 0 & 0 & 1 & 1 & 0 & 1 \\ \hline
1 & 0 & 1 & 0 & 1 & 0 & 0 & 1 \\ \hline
\end{tabular} \
\begin{tabular}{ | c | c | c | c | c | c | c | c |}
\hline
\cellcolor{blue!25}1 & 1 & 1 & 0 & 1 & 1 & 1 & 0 \\ \hline
1 & 0 & 0 & 0 & 1 & 0 & 1 & 1 \\ \hline
1 & 1 & 0 & 1 & 0 & 1 & 1 & 1 \\ \hline
0 & 0 & 0 & 0 & 1 & 1 & 0 & 0 \\ \hline
\end{tabular}
\begin{center}
	Θα πρέπει αν μεταφερθεί ως εξής:\\
	\begin{tabular}{ | c | c | c | c | c | c | c | c | c | c |}
	\hline
	\cellcolor{blue!25}0 & \cellcolor{blue!25}1 & \cellcolor{blue!25}1 & \cellcolor{blue!25}1 & \cellcolor{blue!25}0 & \cellcolor{blue!25}1 & \cellcolor{blue!25}1 & \cellcolor{blue!25}1 & \cellcolor{blue!25}0 & \cellcolor{blue!25}1 \\ \hline
	\cellcolor{blue!25}1 & 1 & 1 & 0 & 1 & 0 & 1 & 0 & 1 & \cellcolor{blue!25}1 \\ \hline
	\cellcolor{blue!25}1 & 1 & 0 & 0 & 0 & 1 & 1 & 1 & 0 & \cellcolor{blue!25}0\\ \hline
	\cellcolor{blue!25}1 & 0 & 1 & 1 & 1 & 1 & 0 & 1 & 0 & \cellcolor{blue!25}0\\ \hline
	\cellcolor{blue!25}0 & 1 & 0 & 1 & 0 & 1 & 1 & 0 & 1 & \cellcolor{blue!25}0\\ \hline
	\cellcolor{blue!25}1 & \cellcolor{blue!25}1 & \cellcolor{blue!25}0 & \cellcolor{blue!25}1 & \cellcolor{blue!25}0 & \cellcolor{blue!25}1 & \cellcolor{blue!25}1 & \cellcolor{blue!25}1 & \cellcolor{blue!25}1 & \cellcolor{blue!25}1\\ \hline
	\end{tabular} \
\end{center}

Αυτό γίνεται μέσω της \lstinline!bboard_to_ext()!:
\begin{lstlisting}[caption={Extend bboard}, escapechar=$, breaklines=true]
__device__
bboard ext_to_bboard(ext_bboard val) {
    bboard res = 0;
    for (int i = 1; i < EXT_HEIGHT - 1; i++) {
        for (int j = 1; j < EXT_WIDTH - 1; j++) {
            if (EXT_BOARD_IS_SET(val, i, j)) SET_BOARD(res, i - 1, j - 1);
        }
    }
    return res;
}

...

ext_bboard res = 0;
for (int i = 0; i < 3; i++) {
    for (int j = 0; j < 3; j++) {
        res |= bboard_to_ext(neighbors[i][j], i, j);
    }
}
\end{lstlisting}

Ο τελικός υπολογισμός γίνεται μέσω της \lstinline!gol()!.
Οι πράξεις που γίνονται στην συνάρτηση βρέθηκαν \href{http://www.onjava.com/pub/a/onjava/2005/02/02/bitsets.html?page=2}{εδώ}.
Ουσιαστικά, υπολογίζονται τα bits που έχουν άθροισμα ζωντανών γειτόνων 3 και 2.

\section{bboard1x64}

Λειτουργεί με παρόμοιο τρόπο με διαφορές στον πίνακα \lstinline!neighbors! και τις bitwise πράξεις που εκτελούνται.
Εδώ, δεν διαχωρίζεται η συνάρτηση ανάλογα με το αν διαιρείται το $N$ με το 8.
